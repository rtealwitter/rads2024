\documentclass{article}
\usepackage[utf8]{inputenc}
\usepackage[a4paper, total={6in, 8in}]{geometry}
\usepackage{amsmath, amsfonts}
\usepackage{hyperref, graphicx}


\title{CSCI 1052 Problem Set 2}
\author{} % TODO: Put your name here
\date{\today}

\begin{document}

\maketitle

\subsection*{Submission Instructions}

Please upload your solutions by
\textbf{5pm Friday January 19, 2024.}
\begin{itemize}
\item You are encouraged to discuss ideas
and work with your classmates. However, you
\textbf{must acknowledge} your collaborators
at the top of each solution on which
you collaborated with others 
and you \textbf{must write} your solutions and code
independently.
\item Your solutions to theory questions must
be typeset in LaTeX or markdown.
I strongly recommend uploading the source LaTeX (found 
\href{https://www.rtealwitter.com/rads2024/psets/pset2.tex}{here})
to Overleaf for editing.
\item I recommend that you write your solutions to coding question in a Jupyter notebook using Google Colab.
\item You should submit your solutions as a \textbf{single PDF} via the assignment on Gradescope. You can enroll in the class using the code GPXX7N.
\item Once you uploaded your solution, \textbf{mark where you answered each part of each question}.
\end{itemize}

\newpage

\section*{Problem 1: Normal Distribution from Darts}

\begin{figure}[h]
	\centering
	\includegraphics{graphics/dartboard.jpeg}
	\caption{A dartboard. Intuitively, the likelihood that a dart hits a particular point should only depend on the distance to the center and rotating the dartboard shouldn't change where darts likely land.}
\end{figure}

In this problem, we will derive the density function of the normal distribution from the example of a dartboard.
Let $f: \mathbb{R}^d \to [0,1]$ be a probability density function that describes the probability a dart lands at the point $(x,y)$.
We want our probability density function $f$ to have two properties:
\begin{enumerate}
    \item \textbf{Radial symmetry:} The probability that a dart lands at a point depends only on the distance between the point and the origin.
    \item \textbf{Independence:} Coordinates are independent e.g., knowing the $x$-coordinate does not give us information about the $y$-coordinate.
\end{enumerate}

From the radial symmetry property, we can conclude that
\begin{align}\label{eq:radial_symmetry}
f(x,y) = f(r) = f(\sqrt{x^2 + y^2})
\end{align}
where $r=\sqrt{x^2+y^2}$ is the distance between the origin and $(x,y)$.

From the independence property, we can conclude that $f(x,y) = g(x) h(y)$ for some functions $g$ and $h$.
Further, the radial symmetry property tells us that $f(x,y) = f(y,x) = g(y) h(x)$ so $g$ and $h$ must be the same function.
That is,
\begin{align}\label{eq:independence}
f(x,y) = g(x) g(y).
\end{align}

\subsection*{Part 1 (.5 points)} 

Up to rescaling, we can assume that $g(0)=1$.
Use this fact and plug in the point $(r,0)$ to conclude that $f$ and $g$ are the same function.

\subsection*{Part 2 (.5 points)}

Define the function $h(x) = f(\sqrt{x})$.
In particular,
\begin{align}\label{eq:define_h}
f(\sqrt{x^2 + y^2}) = f(x) f(y) \Leftrightarrow
h(x^2 + y^2) = h(x^2) h(y^2).
\end{align}
Use this property to show that
\begin{align}\label{eq:h_linear}
h(x_1 + x_2 + \ldots + x_n) = h(x_1) h(x_2) \cdot \ldots \cdot h(x_n).
\end{align}
Then prove that $h(n) = h(1)^n$.

Without loss of generality, let $h(1) = b$.

\subsection*{Part 3 (1 point)}

Let $p$ and $q$ be any integers.
Prove that 
\begin{align}\label{eq:to_prove}
h\left(\frac{p}{q}\right) = b^{\frac{p}{q}}.
\end{align}

\textbf{Hint:} Consider the expression
\begin{align*}
h\left(\frac{p}{q} \cdot q \right).
\end{align*}

As long as $h$ is continuous, conclude that $h(x) = b^x$ for any real number since the rational numbers are dense in the real number line.

\subsection*{Part 4 (1 point)}

You have shown that $h(x) = b^x$. Further, $h(x) = e^{cx}$ for some real number $c$.
Then $f(x) = h(x^2) = e^{c x^2}$.
Recall this implies the density function of $x$ and $y$ is $f(x,y) = f(x) f(y) = e^{c (x^2 + y^2)}$.

For $f$ to be a valid probability density function, what constraint do we have on $c$?
If we wanted to make the distribution more concentrated, how should we change $c$?

Except for the normalization, we have explained every part of the multivariate normal distribution given below
\begin{align*}
f(x,y) = \frac1{\sigma^2 \cdot 2 \pi} \exp\left(-\frac{x^2 + y^2}{2\sigma^2}\right).
\end{align*}
Why is there a normalization?

%\input{solutions/solution2_1.tex}

\newpage

\section*{Problem 2: Johnson-Lindenstrauss for Join Size Estimations}

In class, we showed the Johnson-Lindenstrauss Lemma for preserving the norm of differences.
Consider vectors $\mathbf{x}_1, \ldots, \mathbf{x}_n \in \mathbb{R}^d$.
Let $k = O\left( \frac{\log n}{\epsilon^2} \right)$.
We showed that a random matrix $\mathbf{\Pi} \in \mathbb{R}^{k \times d}$ satisfies
\begin{align}
(1-\epsilon) \| \mathbf{x}_i - \mathbf{x}_j \|_2^2
\leq \| \mathbf{\Pi x}_i - \mathbf{\Pi x}_j \|_2^2
\leq (1+\epsilon) \| \mathbf{x}_i - \mathbf{x}_j \|_2^2
\end{align}
with probability 9/10.

\subsection*{Part 1}
Show that we can use the Johnson-Lindenstrauss Lemma as stated above to show that inner-products are also preserved.
In particular, under the same conditions as above,
\begin{align}
|
\langle \mathbf{x}_i, \mathbf{x}_j \rangle
- \langle \mathbf{\Pi x}_i, \mathbf{\Pi x}_j \rangle
|
\leq \frac12 \epsilon (\|\mathbf{x}_i\|_2^2 + \|\mathbf{x}_j\|_2^2)
\end{align}
with probability 9/10.

\paragraph{Hint 1:} Show that $\| \mathbf{x}_i - \mathbf{x}_j \|_2^2 = \| \mathbf{x}_i \|_2^2 - 2 \langle \mathbf{x}_i, \mathbf{x}_j \rangle + \| \mathbf{x}_j \|_2^2$.

\paragraph{Hint 2:} Apply the Johnson-Lindenstrauss Lemma to the term $\| \mathbf{x}_i + \mathbf{x}_j \|_2^2 - \| \mathbf{x}_i - \mathbf{x}_j \|_2^2$.

\subsection*{Part 2}
One powerful application of sketching is in database applications. For example, a common goal is to estimate the \emph{inner join size} of two tables without performing an actual inner join (which is expensive, as it requires enumerating the keys of the tables).
Formally, consider two sets of unique keys $X = \{x_1, \ldots, x_m\}$ and $Y = \{y_1, \ldots, y_n\}$ which are subsets of $1,2, \ldots, U$. 
Our goal is to estimate $|X\cap Y|$ based on small space compressions of $X$ and $Y$.  

Using your result from Part 1, describe a method based on inner product estimation that constructs independent sketches of $X$ and $Y$ of size  $k = O\left(\frac{\log n}{\epsilon^2}\right)$ and from these sketches can return an estimate $Z$ for $|X\cap Y|$ satisfying
\begin{align*}
	\left|Z - |X\cap Y|\right| \leq \epsilon (|X|+|Y|)
\end{align*}
with probability $9/10$.

%\input{solutions/solution2_2.tex}

\end{document}